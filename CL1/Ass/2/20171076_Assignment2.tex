\documentclass[11pt,a4paper,twocolumn]{article}
\usepackage{fontspec}
\defaultfontfeatures{Mapping=tex-text}
\usepackage{xunicode}
\usepackage{xltxtra}
%\setmainfont{???}
\usepackage{amsmath}
\usepackage{amsfonts}
\usepackage{amssymb}
\usepackage{graphicx}
\usepackage[left=1cm,right=1cm,top=2cm,bottom=2cm]{geometry}

\usepackage[document]{ragged2e}

\font\serif="FreeSerif:script=beng"
\font\bg="FreeSerif:script=beng" at 13pt

\usepackage{xcolor}
\definecolor{hlit}{rgb}{0.5,0,1}
\definecolor{diff}{rgb}{0.1,0.1,0.5}
\definecolor{rlit}{rgb}{1,0,0}

\title{Computational Linguistics - 1, Assignment 2}
\author{Zubair Abid, 20171076}
\date{}

\begin{document}
	\twocolumn[
	\begin{@twocolumnfalse}
    	\maketitle
    	The three tasks of the asssignment:\\
    	\begin{enumerate}
    		\item Comparing Indian POS tagger output on
    		previous assignment data to own output, and
    		analyzing and commenting on the differences.
    		
    		\item Chunk given data with proper labels 
    		(English)
    		
    		\item Chunk Indian language data with proper
    		labels
    	\end{enumerate}
		$ $\\
		$ $\\
    	
  	\end{@twocolumnfalse} 
	]
	
	\section{Compare Indian POS tagger output to own output}	
	Running the test data from Assignment 1 through the 
	LTRC Shallow Parser, analysing and commenting on the
	differences between the two.	
	\subsection{Side-by-side comparison}
	{\bg 
	\begin{enumerate}
		\item Given sentence: 
		\textcolor{diff}{আমার আজ দুপুর থেকেই বেশ ঘুম পেয়েছে}\\
		$ $\\
		Tags (form: word\textcolor{hlit}{/Own tag}
		\textcolor{rlit}{/LTRC}):\\$ $\\
আমার\textcolor{hlit}{/PR\_ PRP}\textcolor{rlit}{/PRP}
আজ\textcolor{hlit}{/N\_ NN}\textcolor{rlit}{/NN}
দুপুর\textcolor{hlit}{/N\_ NN}\textcolor{rlit}{/NN}
থেকেই\textcolor{hlit}{/PSP}\textcolor{rlit}{/untagged}
বেশ\textcolor{hlit}{/RP\_ INTF}\textcolor{rlit}{/INTF}
ঘুম\textcolor{hlit}{/N\_ NN}\textcolor{rlit}{/NN}
পেয়েছে\textcolor{hlit}{/V\_ VM}\textcolor{rlit}{/VM}\\
		$ $\\
		\item Given sentence: \textcolor{diff}{এই কাজ জিনিষটা আমার একদম ভাল লাগে না}\\
		$ $\\
		zTags (form: word\textcolor{hlit}{/Own tag}
		\textcolor{rlit}{/LTRC}):\\$ $\\
এই\textcolor{hlit}{/DM\_ DMD}\textcolor{rlit}{/DEM}
কাজ\textcolor{hlit}{/N\_ NN}\textcolor{rlit}{/NN}
জিনিষটা\textcolor{hlit}{/N\_ NN}\textcolor{rlit}{/NN}
আমার\textcolor{hlit}{/PR\_ PRP}\textcolor{rlit}{/PRP}
একদম\textcolor{hlit}{/RB}\textcolor{rlit}{/RB}
ভাল\textcolor{hlit}{/JJ}\textcolor{rlit}{/JJ}
লাগে\textcolor{hlit}{/V\_ VM\_ VF}\textcolor{rlit}{/VM}
না\textcolor{hlit}{/RP\_ NEG}\textcolor{rlit}{/untagged}\\
		$ $\\
		\item Given sentence: \textcolor{diff}{পরীক্ষাটা একদম জঘন্ন হতে চলছে}\\
		$ $\\
		Tags (form: word\textcolor{hlit}{/Own tag}
		\textcolor{rlit}{/LTRC}):\\$ $\\
পরীক্ষাটা\textcolor{hlit}{/N\_ NN}\textcolor{rlit}{/NN}
একদম\textcolor{hlit}{/RB}\textcolor{rlit}{/RB}
জঘন্ন\textcolor{hlit}{/JJ}\textcolor{rlit}{/JJ}
হতে\textcolor{hlit}{/V\_ AUX}\textcolor{rlit}{/VM}
চলছে\textcolor{hlit}{/V\_ VM\_ VF}\textcolor{rlit}{/VM}\\
		$ $\\
		\item Given sentence: \textcolor{diff}{আমার বাড়িতে একদিন আয়, বেশ মজা হবে}\\
		$ $\\
		Tags (form: word\textcolor{hlit}{/Own tag}
		\textcolor{rlit}{/LTRC}):\\$ $\\
আমার\textcolor{hlit}{/PR\_ PRP}\textcolor{rlit}{/PRP}
বাড়িতে\textcolor{hlit}{/N\_ NN}\textcolor{rlit}{/NN}
একদিন\textcolor{hlit}{/N\_ NN}\textcolor{rlit}{/NN}
আয়\textcolor{hlit}{/N\_ NN}\textcolor{rlit}{/NN}
,\textcolor{hlit}{/,}\textcolor{rlit}{/SYM}
বেশ\textcolor{hlit}{/RP\_ INTF}\textcolor{rlit}{/INTF}
মজা\textcolor{hlit}{/N\_ NN}\textcolor{rlit}{/NN}
হবে\textcolor{hlit}{/V\_ VM\_ VF}\textcolor{rlit}{/VM}\\

	\end{enumerate}			
	}

	\subsection{Analysis}
	Differences in the given data are minimal, primarily
	because most of the words in the bengali lexicon used
	had a single tag, so disambiguation was not a major
	issue. However there do exist some differences, 
	enumerated below:\\
	
	\begin{enumerate}
	\item General vs Specific tags\\
	$ $\\
	Like in {\bg এই\textcolor{hlit}{/DM\_ DMD}\textcolor{rlit}{/DEM}}
	or {\bg লাগে\textcolor{hlit}{/V\_ VM\_ VF}\textcolor{rlit}{/VM}}
	or {\bg হবে\textcolor{hlit}{/V\_ VM\_ VF}\textcolor{rlit}{/VM}}
	or {\bg চলছে\textcolor{hlit}{/V\_ VM\_ VF}\textcolor{rlit}{/VM}}.
	The difference in tag here is the specifity applied.\\
	$ $\\ 
	In the first example the LTRC tagger assigns just a 
	"DEM" tag for Demonstrative instead of the specific
	deictic "DMD" tag applied by my rules.\\
	$ $\\
	In the next three, all the finite main verbs are just
	tagged as main verbs (VM) by the LTRC tagger. 
	
	\item {\bg হতে\textcolor{hlit}{/V\_ AUX}\textcolor{rlit}{/VM}}\\
	$ $\\
	The LTRC tagger considers the bengali {\bg হতে} as a 
	main verb, whereas mine classifies it as an auxillary.
	It relies on the assumption in my ruleset that given
	two verbs in consecutive sequence only one of them is
	marked as a main verb, the other usually gets 
	categorized as auxillary.
	\end{enumerate}			



	\section{English Chunking}
	Labels used: NP, VP, ADJP, ADVP (given in question). \\
	PP is not tagged. \\
	\begin{enumerate}
	\item \textit{She has been absent since last Wednesday.}\\
	$ $\\
	\textcolor{diff}{(( She ))} NP \\ 
	\textcolor{diff}{(( has been ))} VP\\
	\textcolor{diff}{(( absent ))} NP\\
	\textcolor{diff}{(( since ))} \\
	\textcolor{diff}{(( last Wednesday ))} NP\\
	
	
	\item \textit{That movie is getting scarier and scarier.}\\
	$ $\\
	\textcolor{diff}{(( That movie ))} NP\\
	\textcolor{diff}{(( is getting ))} VP\\
	\textcolor{diff}{(( scarier and scarier ))} ADJP\\
	
	\item \textit{What do you do in Japan?}\\
	$ $\\
	\textcolor{diff}{(( What ))} \\
	\textcolor{diff}{(( do ))} \\
	\textcolor{diff}{(( you ))} NP\\
	\textcolor{diff}{(( do ))} VP\\
	\textcolor{diff}{(( in ))} \\
	\textcolor{diff}{(( Japan ))} NP\\
	
	\item \textit{The Handmaid’s Tale is an awesome piece of dystopian 
	fiction.}\\
	$ $\\
	\textcolor{diff}{(( The Handmaid's Tale ))} NP\\
	\textcolor{diff}{(( is ))} \\
	\textcolor{diff}{(( an awesome piece ))} NP\\
	\textcolor{diff}{(( of ))} \\
	\textcolor{diff}{(( dystopian fiction ))} NP\\
	
	\item \textit{OK. Now what?}\\
	$ $\\
	\textcolor{diff}{(( OK ))} ADJP\\
	\textcolor{diff}{(( Now ))} ADVP\\
	\textcolor{diff}{(( What ))} NP\\
	
	\item \textit{Take this medication for pain as often as needed.}\\
	$ $\\
	\textcolor{diff}{(( Take ))} VP\\
	\textcolor{diff}{(( this medication ))} NP\\
	\textcolor{diff}{(( for ))} \\
	\textcolor{diff}{(( pain ))} NP\\
	\textcolor{diff}{(( as often as ))} ADVP\\
	\textcolor{diff}{(( needed ))} ADVP\\
	
	\item \textit{There were people everywhere, covering the 
	roads along the route from the BJP headquarters to the Smriti 
	Sthal from side to side, with security personnel maintaining 
	strict vigil to ensure that nothing goes wrong.}\\
	$ $\\
	\textcolor{diff}{(( There ))} NP\\	
	\textcolor{diff}{(( were ))} VP\\	
	\textcolor{diff}{(( people ))} NP\\
	\textcolor{diff}{(( everywhere ))} ADJP\\
	\textcolor{diff}{(( covering ))} VP\\
	\textcolor{diff}{(( the roads ))} NP\\
	\textcolor{diff}{(( along ))} \\
	\textcolor{diff}{(( the route ))} NP\\
	\textcolor{diff}{(( from ))} \\
	\textcolor{diff}{(( the BJP headquarters ))} NP\\
	\textcolor{diff}{(( to ))} \\
	\textcolor{diff}{(( the Smriti Sthal ))} NP\\
	\textcolor{diff}{(( from ))} \\
	\textcolor{diff}{(( side to side ))} ADVP\\
	\textcolor{diff}{(( with ))} \\
	\textcolor{diff}{(( security personnel ))} NP\\
	\textcolor{diff}{(( maintaining ))} VP\\
	\textcolor{diff}{(( strict vigil ))} NP\\
	\textcolor{diff}{(( to ))} \\
	\textcolor{diff}{(( ensure ))} VP\\
	\textcolor{diff}{(( that nothing ))} NP\\
	\textcolor{diff}{(( goes wrong ))} VP\\
	

	\item \textit{The movie was not too terrible.}\\
	$ $\\
	\textcolor{diff}{(( The movie ))} NP\\
	\textcolor{diff}{(( was ))} VP\\
	\textcolor{diff}{(( not too terrible ))} ADJP\\
	
	
	\end{enumerate}		
	
	\section{Indian Language (Bengali) Chunking}
	
	Labels used NP, VGF, VGNF, VGINF, VGNN, JJP, RBP, NEGP, CCP, FRAGP, BLK.\\

	\begin{enumerate}

	{\bg 

		\item আমার খুব খিদে পেয়েছে \\
		$ $\\
		\textcolor{diff}{(( আমার ))} NP\\
		\textcolor{diff}{(( খুব খিদে ))} NP\\
		\textcolor{diff}{(( পেয়েছে ))} VGF\\
$ $\\
		\item ভেজা মেঝেতে দৌড়োতে হয় না বাবা  \\
		$ $\\
		\textcolor{diff}{(( ভেজা মেঝেতে ))} NP\\
		\textcolor{diff}{(( দৌড়োতে ))} NP\\
		\textcolor{diff}{(( হয় ))} VGF\\
		\textcolor{diff}{(( না ))} NEGP\\
		\textcolor{diff}{(( বাবা ))} NP\\

		\item পরীক্ষায় ভালো না করলে ঘর থেকে বার করে দেব  \\
		$ $\\
		\textcolor{diff}{(( পরীক্ষায় ))} NP\\
		\textcolor{diff}{(( ভালো ))} JJP\\
		\textcolor{diff}{(( না করলে ))} VGINF\\
		\textcolor{diff}{(( ঘর ))} NP\\
		\textcolor{diff}{(( থেকে  ))} \\
		\textcolor{diff}{(( বার ))} NP\\
		\textcolor{diff}{(( করে দেব ))} VGF\\
$ $\\
		\item খাবারে বিরিয়ানি না দিলে বিয়েতে কেউ আসবে না  \\
		$ $\\
		\textcolor{diff}{(( খাবারে ))} NP\\
		\textcolor{diff}{(( বিরিয়ানি ))} NP\\
		\textcolor{diff}{(( না দিলে ))} VGINF\\
		\textcolor{diff}{(( বিয়েতে ))} NP\\
		\textcolor{diff}{(( কেউ ))} NP\\
		\textcolor{diff}{(( আসবে ))} VGF\\
		\textcolor{diff}{(( না ))} NEGP\\
$ $\\
		\item আমার আজ দুপুর থেকেই বেশ ঘুম পেয়েছে  \\
		$ $\\
		\textcolor{diff}{(( আমার ))} NP\\
		\textcolor{diff}{(( আজ ))} NP\\
		\textcolor{diff}{(( দুপুর থেকেই ))} NP\\
		\textcolor{diff}{(( বেশ ঘুম ))} NP\\
		\textcolor{diff}{(( পেয়েছে ))} VGF\\
$ $\\
		\item এই কাজ জিনিষটা আমার একদম ভাল লাগে না  \\
		$ $\\
		\textcolor{diff}{(( এই কাজ ))} NP\\
		\textcolor{diff}{(( জিনিষটা ))} NP\\
		\textcolor{diff}{(( আমার ))} NP\\
		\textcolor{diff}{(( একদম ভাল ))} NP\\
		\textcolor{diff}{(( লাগে ))} VGF\\
		\textcolor{diff}{(( না ))} NEGP\\
$ $\\
		\item পরীক্ষাটা একদম জঘন্ন হতে চলছে  \\
		$ $\\
		\textcolor{diff}{(( পরীক্ষাটা ))} NP\\
		\textcolor{diff}{(( একদম জঘন্ন ))} NP\\
		\textcolor{diff}{(( হতে চলছে ))} VGNF\\
$ $\\		
		\item আমার বাড়িতে একদিন আয়, বেশ মজা হবে  \\
		$ $\\
		\textcolor{diff}{(( আমার বাড়িতে ))} NP\\
		\textcolor{diff}{(( একদিন ))} NP\\
		\textcolor{diff}{(( আয় ))} VP\\
		\textcolor{diff}{(( বেশ ))} JJP\\
		\textcolor{diff}{(( মজা ))} NP\\
		\textcolor{diff}{(( হবে ))} VGF\\
$ $\\
		\item অতিরিক্ত কথা বললে আমি তোদের দুজনকে ক্লাস থেকে বার করে দেব\\
		$ $\\
		\textcolor{diff}{(( অতিরিক্ত কথা ))} NP\\		
		\textcolor{diff}{(( বললে ))} VGINF\\		
		\textcolor{diff}{(( আমি ))} NP\\		
		\textcolor{diff}{(( তোদের দুজনকে ))} NP\\		
		\textcolor{diff}{(( ক্লাস থেকে ))} NP\\		
		\textcolor{diff}{(( বার ))} NP\\		
		\textcolor{diff}{(( করে দেব ))} VGF\\
$ $\\		
		\item সকালে ঘুম থেকে উঠতে গিয়ে আমার দম বেরিয়ে যায় \\
		$ $\\
		\textcolor{diff}{(( সকালে ))} NP\\		
		\textcolor{diff}{(( ঘুম থেকে ))} NP\\		
		\textcolor{diff}{(( উঠতে গিয়ে ))} VGINF\\		
		\textcolor{diff}{(( আমার ))} NP\\		
		\textcolor{diff}{(( দম ))} NP\\		
		\textcolor{diff}{(( বেরিয়ে যায় ))} VGF\\
$ $\\		
		\item কলেজ আমরা একটু বেশি খাবার বাইরে থেকে কিনে 
		খাই কারণ যেটা কলেজ দেয় সেটা বেশ জঘন্য খেতে  \\
		$ $\\
		\textcolor{diff}{(( কলেজ ))} NP\\		
		\textcolor{diff}{(( আমরা ))} NP\\		
		\textcolor{diff}{(( একটু বেশি ))} JJP\\		
		\textcolor{diff}{(( খাবার ))} NP\\		
		\textcolor{diff}{(( বাইরে থেকে ))} NP\\		
		\textcolor{diff}{(( কিনে খাই ))} VGF\\		
		\textcolor{diff}{(( কারণ ))} CCP\\		
		\textcolor{diff}{(( যেটা\ldots সেটা ))} FRAGP\\		
		\textcolor{diff}{(( কলেজ ))} NP\\		
		\textcolor{diff}{(( দেয় ))} VGF\\		
		\textcolor{diff}{(( বেশ জঘন্য ))} RBP\\		
		\textcolor{diff}{(( খেতে ))} VGNF\\		
$ $\\		
		\item এইটা হলো আমার এসসিজিনমেন্টের শেষের সেন্টেন্স  \\
		$ $\\
		\textcolor{diff}{(( এইটা ))} NP\\		
		\textcolor{diff}{(( হলো ))} VGF\\		
		\textcolor{diff}{(( আমার এসসিজিনমেন্টের ))} NP\\		
		\textcolor{diff}{(( শেষের ))} JJP\\		
		\textcolor{diff}{(( সেন্টেন্স ))} NP\\
		
				
		
	}	
	
	\end{enumerate}

\end{document}