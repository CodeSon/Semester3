\documentclass[a4paper, 12pt]{article}

\usepackage{amsmath}

\title{Mathematics III Assignment 1}
\author{Zubair Abid, 20171076}

\begin{document}
\maketitle

\begin{enumerate}

	\item 
	$X$ is a random variable with $PDF$ given by
	
	\[ f(n) = 
		\begin{cases}
			cx^2 & \quad \text{if } x \leq 1 \\
			0 & \quad \text{otherwise}
		\end{cases}
	\]

	\begin{enumerate}
		\item Constant $c$
		\\ \\
		Summation of PDF over domain adds up to 1.\\
		Or, 
		\[
		\int\limits_{-\infty}^{\infty} \mathrm{cx^2}
		\, \mathrm{d}x = 1
		\]
		Since the function returns $0$ everywhere except
		 at $[-1, 1]$, 	we just calculate
		\begin{align*}
			\int\limits_{-1}^{1} \mathrm{cx^2}\, 
			\mathrm{d}x &= 1\\
			c\times \left.\frac{x^3}{3}\right|_{-1}^{1} &= 1\\
			c\times \frac{2}{3} &= 1\\
			c &= 1.5 \text{ (Answer)}
		\end{align*}				
		
		\item $E\left[X\right]$ and $Var\left(X\right)$
		$E[X]$ is 
		\begin{align*}
			\int\limits_{-1}^{1} \mathrm{xcx^2}\,\mathrm{d}x
			&= 1.5 \times \int\limits_{-1}^{1} 
			\mathrm{x^3}\,\mathrm{d}x\\
			&= 1.5 t\times \left.\frac{x^4}{4}\right|_{-1}^{1}\\
			&= 0
		\end{align*}
			
		Now, $Var(X) = E[x^2] - (E[X])^2$, or\\
		\begin{align*}
			Var(X) &= 1.5\times \int\limits_{-1}^{1} 
			\mathrm{x^4}\,\mathrm{d}x - 0\\
			&= 1.5\times \left.\frac{x^5}{5}\right|_{-1}^{1} \\
			&= 1.5 \times 0.4\\
			&= 0.6
		\end{align*}
		
		
		\item $P\left(X \geq \frac{1}{2} \right)$
		\\ \\ 
		Since the function given is a PDF, to get the
		$P(X \geq \frac{1}{2})$, all we need to do is
		integrate $f(x)$ from $\frac{1}{2}$ to $1$\\
		Or, 
		\begin{align*}
			P\left(X \geq \frac{1}{2} \right) &=
			\int\limits_{\frac{1}{2}}^{1} 
			\mathrm{cx^2}\,\mathrm{d}x\\
			&= 1.5 \times \left.\frac{x^3}{3}
			\right|_{\frac{1}{2}}^{1}\\
			&= 1.5 \times \frac{7}{24}\\
			&= 0.4375
		\end{align*}
	\end{enumerate}

	\item Given, the CDF is:
	\[
		F(x) = \frac{x^3 + k}{40} \quad x = 1, 2, 3
	\]
	\begin{enumerate}
		\item Value of k\\
		Since $F(x)$ is a CDF, value of $F(3) = 1$\\
		or 
		\begin{align*}
			\frac{27+k}{40} = 1\\
			k = 13 \quad\text{(Q.E.D)}
		\end{align*}

		\item Find the probability distribution of X\\
		This can be obtained by simple subtraction,
		answer is
		\begin{align*}
			P(X=1) &= \frac{1+13}{40}\\
			&= \frac{7}{20}\\
			P(X=2) &= \frac{21-14}{40}\\
			&= \frac{7}{40}\\
			P(X=3) &= \frac{40-21}{40}\\
			&= \frac{19}{40}
		\end{align*}
		
		\item Given $Var(X) = \frac{259}{320}$,
		 calculate $Var(4X-5)$		
		 \\
		 \\
		 \begin{align*}
		 	{\sigma_{ax+b}}^2 &= a^2 \times {\sigma_x}^2\\
		 	{\sigma_{4x+5}}^2 &= 16 \times {\frac{259}{320}}\\
		 	&= \frac{259}{20}\\
		 	&= 12.95
		 \end{align*}
	\end{enumerate}
	
	\item 
	\begin{align*}
		P \left(\text{First 6 in 2nd throw }\right| 
		\left.\text{First 6 on even throw}\right)
		&= \frac{P\left(\text{2nd throw}\right)\cap
		P\left(\text{even throw}\right)}{P\left(
		\text{even throw}\right)}\\
		&= \frac{\frac{5}{6}\times\frac{1}{6}}
		{\frac{1}{6}\times\left[\sum \frac{5}{6}
		+ \left(\frac{5}{6}\right)^3 + \cdots\right]}\\
		&= \frac{\frac{5}{36}}{\frac{1}{6}\times
		\frac{30}{11}}\\
		&= \frac{\frac{5}{6}}{\frac{30}{11}}\\
		&= \frac{5}{6} \times \frac{11}{30}\\
		&= \frac{11}{36} \left(\text{Answer}\right)
	\end{align*}
	
	\item 
	
	\item 
	
	\begin{enumerate}
		\item
		Sum of a PDF over its given range is 1\\
		Given function:
		\[ f(n) = 
			\begin{cases}
				cx^2 & \quad \text{if } 0 < x < 3 \\
				0 & \quad \text{otherwise}
			\end{cases}
		\]
	
		\begin{align*}
			\int_{0}^{3} \mathrm{cx^2}\,\mathrm{d}x &= 1\\
			c\times\left.\frac{x^3}{3}\right|_{0}^{3} &= 1\\
			c\times 9 &= 1\\
			c &= \frac{1}{9}
		\end{align*}
	
		\item $P(1<X<2)$
		\begin{align*}
			P(1<X<2) &= \frac{1}{9} \times 
			\int_{1}^{2} \mathrm{x^2}\,\mathrm{d}x\\
			&= \frac{1}{9} \times \left.\frac{x^3}{3}
			\right|_{1}^{2}\\
			&= \frac{1}{9} \times \left[ \frac{8}{3}
			- \frac{1}{3}\right]\\
			&= \frac{7}{27}
		\end{align*}
	\end{enumerate}		
	
	\item 
		\begin{enumerate}
		\item
			\begin{align*}
			c\times\int_{-\infty}^{\infty} \mathrm{
			\frac{1}{x^2+1}}\,\mathrm{d}x &= 1\\
			c\times\left.tan^{-1} (x)\right|_{-\infty}^
			{\infty} &= 1\\
			c \times \Pi &= 1\\
			c &= \frac{1}{\Pi}
			\end{align*}
		\item 
			\begin{align*}
			x \in \left(-1, -\sqrt{\frac{1}{3}}\right)
			\cup \left(\sqrt{\frac{1}{3}}, 1\right)\\
			P\left(\frac{1}{3}<x^2<1\right) &= \frac{1}{\Pi}
			\times\left[\int_{-1}^{-\sqrt{\frac{1}{3}}} 
			\mathrm{\frac{1}{1+x^2}}\,\mathrm{d}x + 
			\int_{\sqrt{\frac{1}{3}}}^{1} 
			\mathrm{\frac{1}{1+x^2}}\,\mathrm{d}x\right]\\
			&= \frac{1}{\Pi} \times \left[
			\left.tan^{-1} (x)\right|_{-1}^{-\sqrt{\frac{1}{3}}}
			+ \left.tan^{-1} (x)\right|_{\sqrt{\frac{1}{3}}}^{1}
			\right]\\
			&= \frac{1}{\Pi} \times \left[ -\frac{\Pi}{6}
			+ \frac{\Pi}{4} + \frac{\Pi}{4} - \frac{\Pi}{6} 
			\right]\\
			&= \frac{1}{\Pi} \times \frac{\Pi}{6}\\
			&= \frac{1}{6}
			\end{align*}
		\end{enumerate}

		\item 
		\begin{enumerate}
			\item For values of $x > 0$
			\begin{align*}
				\int_{0}^{x} \mathrm{f(x)}\,\mathrm{d}x &=
				F(x),\quad F(x) = 1 - e^{-2x}\\
				f(x) &= \frac{dF(x)}{dx}\\
				f(x) &= 2e^{-2x}
			\end{align*}
			
			for values of $x < 0$, $f(x) = 0$\\
			Answer:
			\[
				\text{PDF }f(x) = 
				\begin{cases}
					2e^{-2x} \quad \text{if x} > 0\\
					0 \quad \text{otherwise}
				\end{cases}
			\]
			
			\item For $P(X>2)$, we do $F(\infty)-F(2)$
			\begin{align*}
				F(\infty) - F(2) &= 1 - \left(1-e^{-4}\right)\\
				&= e^{-4} \quad\text{(Ans)}
			\end{align*}
			
			\item $P(-3<X\leq4) = P(X\leq4)$
			\begin{align*}
			&= F(4)\\
			&= 1 - e^{-8} \quad\text{(Ans)}
			\end{align*}
		
		\end{enumerate}

	\item 
		\begin{enumerate}
			\item 
			\item
		\end{enumerate}

	\item 
	\item 
	\item 
	\item 
	\item 
	\item 
	\item 
	\item 
	\item 
	
\end{enumerate}

\end{document}